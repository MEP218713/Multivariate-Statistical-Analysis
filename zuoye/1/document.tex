\documentclass[11pt,a4paper]{ctexart}
\usepackage{fontspec}
\defaultfontfeatures{Mapping=tex-text}
\usepackage{xunicode}
\usepackage{xltxtra}
%\setmainfont{???}
\usepackage{amsmath}
\usepackage{amsfonts}
\usepackage{amssymb}
\usepackage{graphicx}
\usepackage{amsthm}
\usepackage{array}
\usepackage{float}   %{H}
\usepackage{booktabs}  %\toprule[1.5pt]
\usepackage[titletoc]{appendix}
%===================%插入代码需要的控制
\usepackage{listings}
\usepackage{xcolor}
\lstset{
	numbers=left, 
	numberstyle= \tiny, 
	keywordstyle= \color{ blue!70},
	commentstyle= \color{red!50!green!50!blue!50}, 
	frame=shadowbox, % 阴影效果
	rulesepcolor= \color{ red!20!green!20!blue!20} ,
	escapeinside=``, % 英文分号中可写入中文
} 
%===================%
\usepackage[left=2cm,right=2cm,top=2cm,bottom=2cm]{geometry}

\newtheorem{theorem}{定理}
\newtheorem{definition}{定义}
\newtheorem*{solution}{解}
\newtheorem{practice}{题}
\newcommand{\Sum}[3][i]{\sum\limits_{#1=#2}^{#3}}
\newcommand{\Int}[2]{\int_{#1}^{#2}}
\newcommand{\Sample}[3][X]{{#1}_{#2},\dotsi ,{#1}_{#3}}
\newcommand{\Samiid}[4][X]{{#1}_{#2},\dotsi ,{#1}_{#3}~iid\backsim {#4}}
\newcommand{\norm}[1]{\left\Vert #1\right\Vert_{\infty}}
\newcommand{\diff}[3]{\frac{\partial^{#3}{#1}}{\partial {#2}^{#3}}}
\newcommand{\abs}[1]{\left| {#1}\right|}
\newcommand{\normdis}[2]{N(#1,{#2}^2)}

\title{应用多元统计分析 作业 (1)}
\author{钟瑜 \quad 222018314210044}
\date{\today}
\begin{document}
\maketitle
\pagestyle{plain}%设置页码
\begin{enumerate}
%================================================================%	
	
\item[1.]利用R中自带的iris数据分别计算三种鸢尾花的sepal长度和宽度的均值和标准差,petal的长度和宽度的均值和标准差;

\textbf{1.1 不分花种}
\begin{lstlisting}[language=r]
> print(head(iris))
     Sepal.Length Sepal.Width Petal.Length Petal.Width Species
1          5.1         3.5          1.4         0.2  setosa
2          4.9         3.0          1.4         0.2  setosa
3          4.7         3.2          1.3         0.2  setosa
4          4.6         3.1          1.5         0.2  setosa
5          5.0         3.6          1.4         0.2  setosa
6          5.4         3.9          1.7         0.4  setosa

> library(psych)
> describe(iris)
vars             n mean sd median trimmed mad min max range skew kurtosis se
Sepal.Length  1 150 5.84 0.83  5.80  5.81 1.04 4.3 7.9  3.6  0.31  -0.61 0.07
Sepal.Width   2 150 3.06 0.44  3.00  3.04 0.44 2.0 4.4  2.4  0.31   0.14 0.04
Petal.Length  3 150 3.76 1.77  4.35  3.76 1.85 1.0 6.9  5.9 -0.27  -1.42 0.14
Petal.Width   4 150 1.20 0.76  1.30  1.18 1.04 0.1 2.5  2.4 -0.10  -1.36 0.06
Species*      5 150 2.00 0.82  2.00  2.00 1.48 1.0 3.0  2.0  0.00  -1.52 0.07
\end{lstlisting}
\textbf{1.2 分花种}
\begin{lstlisting}[language=r]
> vars<- c("Sepal.Length","Sepal.Width","Petal.Length","Petal.Width")
> aggregate(iris[vars], by=list(Species=iris$Species), mean)
Species        Sepal.Length Sepal.Width Petal.Length Petal.Width
1     setosa        5.006       3.428        1.462       0.246
2 versicolor        5.936       2.770        4.260       1.326
3  virginica        6.588       2.974        5.552       2.026

> aggregate(iris[vars], by=list(Species=iris$Species), sd)
Species        Sepal.Length Sepal.Width Petal.Length Petal.Width
1     setosa    0.3524897   0.3790644    0.1736640   0.1053856
2 versicolor    0.5161711   0.3137983    0.4699110   0.1977527
3  virginica    0.6358796   0.3224966    0.5518947   0.2746501
\end{lstlisting}

\hspace*{\fill}
\item[2.]检验三种鸢尾花的sepal长度和宽度,petal的长度和宽度是否存在差异;

\textbf{2.1 散点图的矩阵绘制(1)}
\begin{lstlisting}[language=r]
library(GGally)
ggscatmat(data=iris, columns=1:4, color="Species")
\end{lstlisting}
\begin{figure}[H]
	\centering
	\includegraphics[width=17cm]{1.png}  
	\caption{散点图矩阵}
\end{figure}

\textbf{2.1 散点图的矩阵绘制(2)}
\begin{lstlisting}[language=r]
mycolors<-c("#FF95CA","#FFAF60","#95CACA")
ggpairs(data = iris, aes(color=Species),
columns = c("Sepal.Length","Sepal.Width","Petal.Length",
            "Petal.Width", "Species"))+
scale_fill_manual(values=alpha(mycolors,0.5))
\end{lstlisting}
\begin{figure}[H]
	\centering
	\includegraphics[width=17cm]{2.png}  
	\caption{散点图矩阵}
\end{figure}

\hspace*{\fill}
\item[3.]计算每朵花的sepal长度和宽度的差,petal的长度和宽度的差,并且将计算结果保存到原始数据中,并分别命名为dif\_sepal,dif\_petal;

\begin{lstlisting}[language=r]
dif_Sepal<-iris$Sepal.Length-iris$Sepal.Width
as.matrix(dif_Sepal)
as.data.frame(dif_Sepal)
myiris1<-cbind(iris,dif_Sepal)

dif_Petal<-iris$Petal.Length-iris$Petal.Width
as.matrix(dif_Petal)
as.data.frame(dif_Petal)
myiris<-cbind(myiris1,dif_Petal)
\end{lstlisting}


\item[4.]将第三步中修改后的数据另外保存到G盘,并命名为myiris.csv;
\begin{lstlisting}[language=r]
write.table(myiris,"G:/myiris.csv",
			row.names=FALSE,col.names=TRUE,sep=",")
\end{lstlisting}

\item[5.]用R重新读取myiris.csv。
\begin{lstlisting}[language=r]
> library(readr)
> cancer <- read_csv("G:/myiris.csv")
Parsed with column specification:
cols(
Sepal.Length = col_double(),
Sepal.Width = col_double(),
Petal.Length = col_double(),
Petal.Width = col_double(),
Species = col_character(),
dif_Sepal = col_double(),
dif_Petal = col_double()
)
> View(cancer)
\end{lstlisting}
%=================================================================%
\end{enumerate}
\end{document}